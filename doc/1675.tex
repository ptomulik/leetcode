\documentclass[paper=a4,parskip=half,DIV=12]{leetcode}

\usepackage[T1]{fontenc}
\usepackage[utf8]{inputenc}
\usepackage{courier}
\usepackage{tgtermes,newtxtext,newtxmath}
\usepackage[pdftex,colorlinks,allcolors=blue]{hyperref}
\usepackage{natbib}

\usepackage{amsmath,amsfonts}
\usepackage{tikz}
\usetikzlibrary{arrows,3d,patterns,calc}

\setcitestyle{numbers,square,comma}

\begin{document}

\serietitle{LeetCode contests solutions}
\title{1675. Minimize Deviation in Array~\cite{leetcode:1675}}
\subtitle{}
\author{Paweł Tomulik}
\date{2024-06-19}
\maketitle

\section{Description}
\label{sec:description}

You are given an array \texttt{nums} of $N$ positive integers.

You can perform two types of operations on any element of the array any number
of times:

\begin{itemize}
  \item If the element is \textbf{even}, \textbf{divide} it by $2$. For example, if
    the array is \texttt{[1,2,3,4]}, then you can do this operation on the
    last element, and the array will be \texttt{[1,2,3,2]}.
  \item
    If the element is \textbf{odd}, \textbf{multiply} it by $2$. For example,
    if the array is \texttt{[1,2,3,4]}, then you can do this operation on the first
    element, and the array will be \texttt{[2,2,3,4]}.
\end{itemize}
The deviation of the array is the maximum difference between any two elements
in the array.

Return the minimum deviation the array can have after performing some number of
operations.

\section{Solution}
\label{sec:solution}

\subsection{Complexity}
\label{sec:complexity}

$\mathcal{O}(N \log{(N)})$ or $\mathcal{O}(N)$, depending on the distribution of numbers in \texttt{nums}.

\subsection{Solution description}
\label{sec:solution-description}

So, we have a sequence of natural numbers $\mathrm{nums} = ( a_i )_{i=1,\dots,
N}, \; a_i \in \mathbb{N}$, which can be expressed as
%%%%%%%%%%%%%%%%%%%%%%%%%%%%%%%%%%%%%%%%%%%%%%%%%%%%%%%%%%%%%%%%%%%%%%%%%%%%%%
\begin{equation}
  a_i = m_i 2^{\alpha_i}, \;\; m_i \text{ odd}.
  \label{eq:DLHTR}
\end{equation}
%%%%%%%%%%%%%%%%%%%%%%%%%%%%%%%%%%%%%%%%%%%%%%%%%%%%%%%%%%%%%%%%%%%%%%%%%%%%%%
We are allowed to transform that sequence to $( b_i )_{i=1,\dots,N}$
following the rules prescribed in section~\ref{sec:description}:
%%%%%%%%%%%%%%%%%%%%%%%%%%%%%%%%%%%%%%%%%%%%%%%%%%%%%%%%%%%%%%%%%%%%%%%%%%%%%%
\begin{equation}
  \begin{aligned}
    b_i = m_i 2^{\beta_i}, && \beta_i \in \{ 0, 1, \dots, \max{(1, \alpha_i)} \}.
  \end{aligned}
  \label{eq:3FP4V}
\end{equation}
%%%%%%%%%%%%%%%%%%%%%%%%%%%%%%%%%%%%%%%%%%%%%%%%%%%%%%%%%%%%%%%%%%%%%%%%%%%%%%
Let's denote
%%%%%%%%%%%%%%%%%%%%%%%%%%%%%%%%%%%%%%%%%%%%%%%%%%%%%%%%%%%%%%%%%%%%%%%%%%%%%%
\begin{equation}
  \begin{aligned}
    b_{min} = \min_i b_i, && b_{max} = \max_i b_i.
  \end{aligned}
  \label{eq:JB95I}
\end{equation}
%%%%%%%%%%%%%%%%%%%%%%%%%%%%%%%%%%%%%%%%%%%%%%%%%%%%%%%%%%%%%%%%%%%%%%%%%%%%%%

We're asked to find $\sigma_{min} = \min{(b_{max} - b_{min})}$ over all
sequences $( b_i )$ satisfying \eqref{eq:3FP4V}.

As seen, all our sequences are generated out of a prescribed sequence of
\textbf{odd} numbers $( m_i )_{i=1,\dots,N}$. The sequence $(m_i)$ may be
constructed by dividing all even numbers from $(a_i)$ multiple times, until
they all become odd. It is not necessarily the optimal sequence, we're
searching for, but it appears to be a~useful utility for further analysis.

Let
%%%%%%%%%%%%%%%%%%%%%%%%%%%%%%%%%%%%%%%%%%%%%%%%%%%%%%%%%%%%%%%%%%%%%%%%%%%%%%
\begin{equation}
  \begin{aligned}
    & m_{min} = \min_i m_i, &
    & m_{max} = \max_i m_i.&
  \end{aligned}
  \label{eq:7M3YV}
\end{equation}
%%%%%%%%%%%%%%%%%%%%%%%%%%%%%%%%%%%%%%%%%%%%%%%%%%%%%%%%%%%%%%%%%%%%%%%%%%%%%%%%
then we have the following upper bound for $\sigma_{min}$
%%%%%%%%%%%%%%%%%%%%%%%%%%%%%%%%%%%%%%%%%%%%%%%%%%%%%%%%%%%%%%%%%%%%%%%%%%%%%%%%
\begin{equation}
  \sigma_{min} \le m_{max} - m_{min}
  \label{eq:66HJL}
\end{equation}
%%%%%%%%%%%%%%%%%%%%%%%%%%%%%%%%%%%%%%%%%%%%%%%%%%%%%%%%%%%%%%%%%%%%%%%%%%%%%%%%

In the following discussion we'll rather operate on sets $\{ a_i \}_{i =
1,\dots, A}$, $\{ b_i \}_{i = 1, \dots, B}$, and $\{ m_i \}_{i = 1, \dots M}$
instead of sequences, as the order of sequence elements is irrelevant, and only
distinct values of numbers found in $(a_i)$ matters. Note that cardinalities of
these sets follow $M \le B \le A$.

\subsubsection{Solutions for $3 m_{min} > 2 m_{max}$}
\label{sec:ZCP3Z}

If the range $[m_{min}, m_{max}]$ is far enough from $0$, $\sigma_{min}$ may be
found in~$\mathcal{O}(N)$. We'll prove that
%%%%%%%%%%%%%%%%%%%%%%%%%%%%%%%%%%%%%%%%%%%%%%%%%%%%%%%%%%%%%%%%%%%%%%%%%%%%%%%%
\begin{equation}
  3 m_{min} > 2 m_{max} \implies \sigma_{min} = m_{max} - m_{min}.
  \label{eq:FX72R}
\end{equation}
%%%%%%%%%%%%%%%%%%%%%%%%%%%%%%%%%%%%%%%%%%%%%%%%%%%%%%%%%%%%%%%%%%%%%%%%%%%%%%%%
Proof: suppose, there exists another distribution $\{b_i\}$ of numbers derived
from $\{a_i\}$ which yields $\sigma_{min} = b_{max} - b_{min} < m_{max} -
m_{min}$. If it's true, then we shall be able to find it by just multiplying
numbers from $\{m_i\}$. To ``shrink'' the set, we have to deal with $m_{min}$
and/or $m_{max}$ in the first place. If we multiply $m_{min}$, then the new
deviation is $\sigma \ge 2 m_{min} - m_{max} = 3 m_{min} - m_{min} - m_{max} > 2
m_{max} - m_{min} - m_{max} = m_{max} - m_{min}$. So, even if the set $\{m_i\}$
consists only of $m_{min}$ and $m_{max}$ (in which case $\sigma = 2 m_{min} -
m_{max}$), the new deviation is still $\sigma > m_{max} - m_{min}$. Multiplying
$m_{max}$ alone yields $\sigma \ge 2 m_{max} - m_{min} > m_{max} - m_{min}$.
Multiplying both numbers yields $\sigma \ge 2 (m_{max} - m_{min}) > m_{max} -
m_{min}$. It all means, that the set $\{b_i\}$ with $\sigma < m_{max} - m_{min}$
does not exist.

\textbf{Corollary}: if $3 m_{min} > 2 m_{min}$, the \textbf{$\sigma_{min}$ is
found in~$\mathcal{O}(N)$} (the cost of finding $m_{min}$ and $m_{max}$).

\subsubsection{Solutions for $2 m_{min} > m_{max} \wedge 3 m_{min} < 2 m_{max}$}
\label{sec:6GNBB}

For this case we have
%%%%%%%%%%%%%%%%%%%%%%%%%%%%%%%%%%%%%%%%%%%%%%%%%%%%%%%%%%%%%%%%%%%%%%%%%%%%%%%%
\begin{equation}
  2 m_{min} - m_{max} \le \sigma_{min} \le m_{max} - m_{min}
  \label{eq:9W8Z2}
\end{equation}
%%%%%%%%%%%%%%%%%%%%%%%%%%%%%%%%%%%%%%%%%%%%%%%%%%%%%%%%%%%%%%%%%%%%%%%%%%%%%%%%

Take any value $m \in [m_{min}, m_{max}]$ (not necessarily from $\{m_i\}$).
If~$m$ is \textbf{even} and $m < 2 m_{min}$, then $m$, \textbf{has no
representative} in $\{a_i\}$, that is $m\cdot 2^k \notin \{a_i\}$, for all $k
\in \{ 0,1,\dots \}$. Especially $m \notin \{a_i\}$.

Proof: suppose that $m$ is present in $\{a_i\}$. It is even, so it must
contribute to $\{m_i\}$. Once divided, it yields $\frac{a_i}{2} < m_{min}$ and,
by definition of $m_{min}$: $m_{min} \le \frac{a_i}{2}$, which leads to
a~contradiction $m_{min} < m_{min}$.

\textbf{Corollary}: if $2 m_{min} > m_{max}$, then \textbf{none of even
numbers} $m \in (m_{min}, m_{max})$ \textbf{appear} in $\{a_i\}$.

\textbf{Corollary}: if $2 m_{min} > m_{max}$, then \textbf{all $m_i$'s can be
multiplied} when building the optimal set $\{b_i\}$.

Suppose that $2 m_{min} > m_{max}$ and $3 m_{min} < 2 m_{max}$, for example
$3 m_{min} = 2 m_{max} - k$, $k$ odd. Then we see that $2 m_{min} - m_{max} =
3 m_{min} - m_{min} - m_{max} = 2 m_{max} - k - m_{min} - m_{max} = m_{max} - m_{min} - k$.
If, for example, $\{ m_i \} = \{ m_{min}, m_{max} \}$ (no intermediate values
$m_i$ between $m_{min}$ and $m_{max}$), then $\sigma_{min} = m_{max} - m_{min} - k$.
In general, however $m_{max} - m_{min} - k \le \sigma_{min} \le m_{max} -
m_{min}$. The solution can be found in $\mathcal{O}(N \log{N})$ with
a~procedure $\mathrm{mindev}(\{ m_i \}, m_{max})$ described
in~section~\ref{sec:WTM0T}.

\subsubsection{Solutions for $2 m_{min} < m_{max}$}
\label{sec:2SBRZ}

For this case we have
%%%%%%%%%%%%%%%%%%%%%%%%%%%%%%%%%%%%%%%%%%%%%%%%%%%%%%%%%%%%%%%%%%%%%%%%%%%%%%%%
\begin{equation}
  2 \le \sigma_{min} \le m_{max} - m_{min}
  \label{eq:0Y3X2}
\end{equation}
%%%%%%%%%%%%%%%%%%%%%%%%%%%%%%%%%%%%%%%%%%%%%%%%%%%%%%%%%%%%%%%%%%%%%%%%%%%%%%%%

We've already proven, that all ``obstacles`` (even numbers $o_i$, not allowed
to be further multiplied) are~$\ge 2 m_{min}$. Let's define the ``obstacles''
more formally
%%%%%%%%%%%%%%%%%%%%%%%%%%%%%%%%%%%%%%%%%%%%%%%%%%%%%%%%%%%%%%%%%%%%%%%%%%%%%%%%
\begin{equation}
  \{ o_i \} = \{ a_i: a_i \text{ even } \wedge a_i < m_{max} \} \cup
              \{ 2 a_i: a_i \text{ odd } \wedge  2 a_i < m_{max} \}
  \label{eq:SDKKJ}
\end{equation}
%%%%%%%%%%%%%%%%%%%%%%%%%%%%%%%%%%%%%%%%%%%%%%%%%%%%%%%%%%%%%%%%%%%%%%%%%%%%%%%%
Let
%%%%%%%%%%%%%%%%%%%%%%%%%%%%%%%%%%%%%%%%%%%%%%%%%%%%%%%%%%%%%%%%%%%%%%%%%%%%%%%%
\begin{equation}
  o_{min} = \min \{o_i\},
  \label{eq:T13ET}
\end{equation}
%%%%%%%%%%%%%%%%%%%%%%%%%%%%%%%%%%%%%%%%%%%%%%%%%%%%%%%%%%%%%%%%%%%%%%%%%%%%%%%%
then $b_{min} \le o_{min}$ and $2 m_{min} \le o_{min}$.

All $m_i: 2 m_{min} \le 2 m_i < m_{max}$ may be multiplied without worrying
that this operation would enlarge~$\sigma$. All the multiplied numbers will
fall into $[2 m_{min}, m_{max})$. We need, however, to the careful, as some
resultant numbers may fall into $[2 m_{min}, o_{min})$ -- not covered by
$\{o_i\}$. These numbers must still be traced. All others may be forgotten,
as they are now ``hidden'' behind $o_{min}$.

Let's make a new set $\{ c_i \} = \{ m_i: 2 m_{max} \le 2 m_i \wedge m_i \le o_{min} \}
\cup \{ 2 m_i: 2 m_i \le o_{min} \}$. The solution may be found in
$\mathcal{O}(N \log N)$ with the procedure $\mathrm{mindev}(\{ c_i \}, m_{max})$
from section~\ref{sec:WTM0T}.

\subsection{Procedure \texttt{mindev(cset, mmax)}}
\label{sec:WTM0T}

The procedure takes a set of numbers $\{ c_i: c_i < c_{i+1} \}_{i=1,\dots,C}$,
derived from $\{ m_i \}$, and returns the minimum deviation~$\sigma_{min}$. It
searches for minimum among the deviations of consecutive sets
$\{ c_1, \dots, c_C, m_{max} \}$, $\{ c_2, \dots, c_C, m_{max}, 2 c_1\}$,
\dots, $\{ c_C, m_{max}, 2 c_1, \dots, 2 c_{C-1} \}$. These sets are not
constructed explicitly by the implementation, it only maintains their minima
and maxima.

The time complexity of~\texttt{mindev} is $\mathcal{O}(C),\; C \le N$, but
constructing $\{ c_i \}$ costs $\mathcal{O}(N \log N)$.

\bibliographystyle{unsrtnat}
\bibliography{leetcode}

\end{document}

% vim: set syntax=tex tabstop=2 shiftwidth=2 expandtab spell spelllang=en:
