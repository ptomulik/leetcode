\documentclass[paper=a4,parskip=half,DIV=12]{leetcode}

\usepackage[T1]{fontenc}
\usepackage[utf8]{inputenc}
\usepackage{courier}
\usepackage{tgtermes,newtxtext,newtxmath}
\usepackage[pdftex,colorlinks,allcolors=blue]{hyperref}
\usepackage{subcaption}
\usepackage{natbib}
\usepackage{float}

\usepackage{amsmath,amsfonts}
\usepackage{tikz}
\usetikzlibrary{arrows,calc,graphs,matrix,quotes,positioning,arrows.meta}

\setcitestyle{numbers,square,comma}

\begin{document}

\serietitle{LeetCode contests solutions}
\title{1601. Maximum Number of Achievable Transfer Requests~\cite{leetcode:1601}}
\subtitle{}
\author{Paweł Tomulik}
\date{\today}
\maketitle

\section{Description}
\label{sec:description}

We have $n$ buildings numbered from $0$ to $n - 1$. Each building has a~number
of employees. It's transfer season, and some employees want to change the
building they reside in.

You are given an array \texttt{requests} where \texttt{requests[i] = [from$_i$,
to$_i$]} represents an employee's request to transfer from building
\texttt{from$_i$} to building \texttt{to$_i$}.

All buildings are full, so a list of requests is achievable only if for each
building, the net change in employee transfers is zero. This means the number
of employees leaving is equal to the number of employees moving in. For example
if $n = 3$ and two employees are leaving building $0$, one is leaving building
$1$, and one is leaving building $2$, there should be two employees moving to
building $0$, one employee moving to building $1$, and one employee moving to
building $2$.

Return the \textbf{maximum number of achievable requests}.

\section{Solution}
\label{sec:solution}

This is a~variant of network flow optimization~\cite{ahuja1993network} --
the maximum circulation problem.

\subsection{Complexity}
\label{sec:complexity}

To be determined, but it shall be not worse then $\mathcal{O}(|V|^3 + |A|)$.
Utilizes Tarjan algorithm for strongly connected components ($\mathcal{O}(|V| + |A|)$)
at preprocessing, Dijkstra's shortest path algorithm in main loop loop
(up to $\mathcal{O}(|V|)$ times, I guess) and an iteration which, I guess,
costs $\mathcal{O}(|A|)$.

\subsection{Solution description}
\label{sec:solution-description}

Our problem can be modelled using a~directed graph $G = (V, A)$ with a~set of
nodes $V = \{0, \dots, n-1 \}$ and a~set $A \subseteq V\times V$ of arcs
(ordered pairs $(i,j) \in A$). Nodes represent buildings, arcs correspond to
transfer requests. We~assign a~positive integer $u_{ij}$ to each arc $(i, j)
\in A$
%%%%%%%%%%%%%%%%%%%%%%%%%%%%%%%%%%%%%%%%%%%%%%%%%%%%%%%%%%%%%%%%%%%%%%%%%%%%%
\begin{equation}
    u_{ij} \in \mathbb{N}, \; (i, j) \in A,
    \label{eq:49EZ8}
\end{equation}
%%%%%%%%%%%%%%%%%%%%%%%%%%%%%%%%%%%%%%%%%%%%%%%%%%%%%%%%%%%%%%%%%%%%%%%%%%%%%
and set it to the~number of transfer requests from building $i$ to $j$. Let
$r_{ij}$ be the number of achievable requests from $i$ to $j$. In terms of
network flow $r_{ij}: 0 \le r_{ij} \le u_{ij}$ is a network flow over
an individual arc~$(i,j)$ and $u_{ij}$ is the capacity of the arc. The maximum
circulation problem can be stated as follows.

\paragraph{Maximum circulation problem.} Let $r_{ij}$ be a~flow along arc
$(i,j)$ and $r = \left\lbrace r_{ij} \right\rbrace$. Maximize
%%%%%%%%%%%%%%%%%%%%%%%%%%%%%%%%%%%%%%%%%%%%%%%%%%%%%%%%%%%%%%%%%%%%%%%%%%%%%
\begin{subequations}
  \begin{equation}
    z(r) = \sum_{(i,j) \in A} r_{ij},
    \label{eq:UDSSB}
  \end{equation}
  subject to the constraints
  \begin{align}
    \sum_{j: (j,i) \in A} r_{ji} - \sum_{j: (i,j) \in A} r_{ij} = 0, &
    && \text{ for all } i \in V
    && \text{(flow through node must be preserved)},
    \label{eq:N3KEB}
    \\
    0 \le r_{ij} \le u_{ij}, &
    && \text{ for all } (i, j) \in A
    && \text{(arcs must not be overflowed)}.
    \label{eq:JV5KC}
  \end{align}
  \label{eq:YA13L}
\end{subequations}
%%%%%%%%%%%%%%%%%%%%%%%%%%%%%%%%%%%%%%%%%%%%%%%%%%%%%%%%%%%%%%%%%%%%%%%%%%%%%

One possible approach is to transform the above problem to a corresponding
minimum cost flow problem, which is known to have standard solution algorithms.
If we flood our network by pushing $u_{ij}$ through all arcs $(i,j) \in A$,
the node flow preservation constraints \eqref{eq:N3KEB} get violated imposing
imbalances $e_i$ in some nodes. The sum of arc flows in the flooded network is
then
%%%%%%%%%%%%%%%%%%%%%%%%%%%%%%%%%%%%%%%%%%%%%%%%%%%%%%%%%%%%%%%%%%%%%%%%%%%%%
\begin{subequations}
  \begin{equation}
    z(u) = \sum_{(i,j) \in A} u_{ij},
    \label{eq:I4AB9}
  \end{equation}
   with
  \begin{align}
    \sum_{j: (j,i) \in A} u_{ji} - \sum_{j: (i,j) \in A} u_{ij} = e_i, &
    && \text{ for all } i \in V
    && \text{(imbalance in node)}.
    \label{eq:T1XA8}
  \end{align}
  \label{eq:D2Y3O}
\end{subequations}
%%%%%%%%%%%%%%%%%%%%%%%%%%%%%%%%%%%%%%%%%%%%%%%%%%%%%%%%%%%%%%%%%%%%%%%%%%%%%

Nodes with $e_i = 0$ will be called {\em balanced}, with $e_i < 0$ -- {\em
deficit}, and with $e_i > 0$ -- {\em excess}.

The flow $u = \left\lbrace u_{ij} \right\rbrace$ may be decomposed onto $r =
\left\lbrace r_{ij}\right\rbrace$ and~$x = \left\lbrace x_{ij} \right\rbrace$
as follows
%%%%%%%%%%%%%%%%%%%%%%%%%%%%%%%%%%%%%%%%%%%%%%%%%%%%%%%%%%%%%%%%%%%%%%%%%%%%%
\begin{subequations}
  \begin{equation}
    z(u) = z(r) + z(x)
    \label{eq:6M1IM}
  \end{equation}
  with the following properties
  \begin{align}
    \sum_{j: (j,i) \in A} r_{ji} - \sum_{j: (i,j) \in A} r_{ij} = 0, &
    && \text{ for all } i \in V
    && \text{(circular flow)},
    \label{eq:7FT2U}
    \\
    \sum_{j: (j,i) \in A} x_{ji} - \sum_{j: (i,j) \in A} x_{ij} = e_i, &
    && \text{ for all } i \in V
    && \text{(remaining flow)},
    \label{eq:VXBVY}
    \\
    r_{ij} \ge 0 \land x_{ij} \ge 0, &
    && \text{ for all } (i, j) \in A
    && \text{(flow components are non-negative)},
    \label{eq:WDOJ8}
    \\
    r_{ij} + x_{ij} = u_{ij}, &
    && \text{ for all } (i, j) \in A
    && \text{(decomposition of single arc flow)}.
    \label{eq:K3II1}
  \end{align}
  \label{eq:KSU4J}
\end{subequations}
%%%%%%%%%%%%%%%%%%%%%%%%%%%%%%%%%%%%%%%%%%%%%%%%%%%%%%%%%%%%%%%%%%%%%%%%%%%%%

The decomposition \eqref{eq:KSU4J} is non-unique. For example, $(x, r) = (u, 0)$
is feasible for any input data $(G,u)$, but it's almost never optimal ($z(r) = 0$).
Among all the feasible decompositions \eqref{eq:KSU4J}, there is at least one
optimal $(x^{*}, r^{*})$ having maximum circulation $z(r^{*})$. But the same
decomposition will also have minimum $z(x^{*})$ -- see~\eqref{eq:6M1IM}. What
follows is, that searching for maximum circulation $z(r^{*})$ is equivalent to
searching for a~flow $x^{*}$ of minimum cost $z(x^{*})$. The problem might be
now restated as follows.

\paragraph{Minimum cost flow.} Minimize the flow cost
%%%%%%%%%%%%%%%%%%%%%%%%%%%%%%%%%%%%%%%%%%%%%%%%%%%%%%%%%%%%%%%%%%%%%%%%%%%%%
\begin{subequations}
  \begin{equation}
    z(x) = \sum_{(i,j) \in A}{x_{ij}}
    \label{eq:0NZPS}
  \end{equation}
  subject to
  \begin{align}
    \sum_{j: (j,i) \in A}{x_{ji}} - \sum_{j: (i,j) \in A}{x_{ij}} = e_i, &
    && \text{ for all } i \in V, &&
    \label{eq:NMX02}
    \\
    0 \le x_{ij} \le u_{ij}, &
    && \text{ for all } (i,j) \in A. &&
    \label{eq:O0QTB}
  \end{align}
  \label{eq:A20W8}
\end{subequations}
%%%%%%%%%%%%%%%%%%%%%%%%%%%%%%%%%%%%%%%%%%%%%%%%%%%%%%%%%%%%%%%%%%%%%%%%%%%%%
This is a standard minimum cost flow problem with arc costs $c_{ij} = 1$, for
all $(i,j) \in A$ and node supplies $b_i = -e_i$.

\paragraph{Path and cycle flow.} There is a~flow decomposition
theorem~\cite{ahuja1993network} saying that an arc flow (say, $u$)
may be decomposed onto path and cycle flow ($x$ and $r$ in our case: $u = x + r$).
The decomposition is non-unique -- in general, there are multiple ways to
decompose given arc flow $u$, yielding different values of $(z(x), z(r))$. In
the maximum circulation problem, we look for a~decomposition $(x^{*},r^{*})$
having maximum cycle flow $z(r^{*})$, or, alternatively, having minimum cost
$z(x^{*})$ of its path flow. The path flow $x$ can be expressed as
a~superposition of prescribed flows $f(P_{kl})$ along individual paths $P_{kl}
\subseteq A$ connecting deficit nodes $k \in D = \left\lbrace i: e_i < 0
\right\rbrace$ to excess nodes $l \in E = \left\lbrace i: e_i > 0\right\rbrace$,
that is
%%%%%%%%%%%%%%%%%%%%%%%%%%%%%%%%%%%%%%%%%%%%%%%%%%%%%%%%%%%%%%%%%%%%%%%%%%%%%
\begin{equation}
  x_{ij} = \sum_{P_{kl}: (i,j) \in P_{kl}}{f(P_{kl})}.
  \label{eq:AIKU9}
\end{equation}
%%%%%%%%%%%%%%%%%%%%%%%%%%%%%%%%%%%%%%%%%%%%%%%%%%%%%%%%%%%%%%%%%%%%%%%%%%%%%

\paragraph{Algorithm outline.}
The proposed method is similar to the {\em consecutive shortest paths}
algorithm~\cite{ahuja1993network}. We'll identify and subtract path flows
$f(P_{kl})$ along paths $P_{kl}$ connecting deficit nodes to excess nodes.
The contribution of a~single path $P_{kl}$ to the cost $z(x)$ is $f(P_{kl})
\cdot |P_{kl}|$, where $|P_{kl}|$ is the length of $P_{kl}$. The total cost can
thus be expressed as
%%%%%%%%%%%%%%%%%%%%%%%%%%%%%%%%%%%%%%%%%%%%%%%%%%%%%%%%%%%%%%%%%%%%%%%%%%%%%
\begin{equation}
  z(x) = \sum_{P_{kl} \in \mathcal{P}}{f(P_{kl}) \cdot |P_{kl}|},
  \label{eq:0SQMY}
\end{equation}
%%%%%%%%%%%%%%%%%%%%%%%%%%%%%%%%%%%%%%%%%%%%%%%%%%%%%%%%%%%%%%%%%%%%%%%%%%%%%
where $\mathcal{P}$ denotes a~set of all paths selected to carry the path flow.
To find a~path flow $x$ of minimum cost, each $P_{kl} \in \mathcal{P}$ must be
a~shortest path from $k$ to $l$. For any fixed set of values $f(P_{kl})$,
taking longer paths would inevitably generate higher cost $z(x)$ (which is
evident from \eqref{eq:0SQMY}). On the other hand, as the path flow $x$ models
the component of flood $u$ entirely responsible for imbalances $e_i$, it must
then hold
%%%%%%%%%%%%%%%%%%%%%%%%%%%%%%%%%%%%%%%%%%%%%%%%%%%%%%%%%%%%%%%%%%%%%%%%%%%%%
\begin{equation}
  \sum_{P_{kl} \in \mathcal{P}}{f(P_{kl})} = \sum_{i \in D}{-e_i} = \sum_{i \in E}{e_i},
  \label{eq:QDPFI}
\end{equation}
%%%%%%%%%%%%%%%%%%%%%%%%%%%%%%%%%%%%%%%%%%%%%%%%%%%%%%%%%%%%%%%%%%%%%%%%%%%%%
with deficit nodes $D = \left\lbrace i: e_i < 0\right\rbrace$ and excess nodes
$E = \left\lbrace i: e_i > 0\right\rbrace$. Moreover, the sum of path flows
originating from any deficit node $k \in D$ must be
%%%%%%%%%%%%%%%%%%%%%%%%%%%%%%%%%%%%%%%%%%%%%%%%%%%%%%%%%%%%%%%%%%%%%%%%%%%%%
\begin{equation}
  \begin{aligned}
    & \sum_{l: P_{kl} \in \mathcal{P}}{f(P_{kl})} = -e_k, && \text{ for all } k \in D, &
  \end{aligned}
  \label{eq:I9S7G}
\end{equation}
%%%%%%%%%%%%%%%%%%%%%%%%%%%%%%%%%%%%%%%%%%%%%%%%%%%%%%%%%%%%%%%%%%%%%%%%%%%%%
and the sum of path flows sinking to any excess node $l \in E$ must be
%%%%%%%%%%%%%%%%%%%%%%%%%%%%%%%%%%%%%%%%%%%%%%%%%%%%%%%%%%%%%%%%%%%%%%%%%%%%%
\begin{equation}
  \begin{aligned}
    & \sum_{k: P_{kl} \in \mathcal{P}}{f(P_{kl})} = e_l, && \text{ for all } l \in E. &
  \end{aligned}
  \label{eq:MW3RR}
\end{equation}
%%%%%%%%%%%%%%%%%%%%%%%%%%%%%%%%%%%%%%%%%%%%%%%%%%%%%%%%%%%%%%%%%%%%%%%%%%%%%

The flooded network corresponds to a~residual network $G(x^0)$ with $x^0 = 0$
and residual capacities $r^0 = u$\footnote{For our purposes, we define
a~residual network $G(x^{\eta})$ w.r.t pseudo-flow $x^{\eta}$ as a graph $G$
having arc capacities $r_{ij} = u_{ij} - x^{\eta}_{ij}$.}. We treat the
circular pseudo-flow $r$ as residual capacities and will modify it from
iteration to iteration. At each iteration $\eta$, a~pair $(k,l) \in D \times E$
of imbalanced nodes is selected and a~flow along a~shortest path $P_{kl}$ from
$k$ to $l$ is determined
%%%%%%%%%%%%%%%%%%%%%%%%%%%%%%%%%%%%%%%%%%%%%%%%%%%%%%%%%%%%%%%%%%%%%%%%%%%%%
\begin{equation}
  f(P_{kl}) = \min_{(i,j) \in P_{kl}}{\left\lbrace -e_k, e_l, r^{\eta}_{ij} \right\rbrace}.
  \label{eq:WCR8C}
\end{equation}
%%%%%%%%%%%%%%%%%%%%%%%%%%%%%%%%%%%%%%%%%%%%%%%%%%%%%%%%%%%%%%%%%%%%%%%%%%%%%
The flow $f(P_{kl})$ is then added to $x$ and subtracted from $r$ (network
update):
%%%%%%%%%%%%%%%%%%%%%%%%%%%%%%%%%%%%%%%%%%%%%%%%%%%%%%%%%%%%%%%%%%%%%%%%%%%%%
\begin{equation}
  \begin{aligned}
    & x^{\eta+1}_{ij} = x^{\eta}_{ij} + f(P_{kl}) && \text{ for all } (i,j) \in P_{kl}, &
    \\
    & r^{\eta+1}_{ij} = r^{\eta}_{ij} - f(P_{kl}) && \text{ for all } (i,j) \in P_{kl}, &
    \\
    & e^{\eta+1}_k = e^{\eta}_k + f(P_{kl}), &
    & \text{ also, if } e^{\eta+1}_k = 0, \text{ then remove }k \text{ from } D, &
    \\
    & e^{\eta+1}_l = e^{\eta}_k - f(P_{kl}), &
    & \text{ also, if } e^{\eta+1}_l = 0, \text{ then remove }l \text{ from } E. &
  \end{aligned}
  \label{eq:WTLZT}
\end{equation}
%%%%%%%%%%%%%%%%%%%%%%%%%%%%%%%%%%%%%%%%%%%%%%%%%%%%%%%%%%%%%%%%%%%%%%%%%%%%%
Initial values for the iteration are
%%%%%%%%%%%%%%%%%%%%%%%%%%%%%%%%%%%%%%%%%%%%%%%%%%%%%%%%%%%%%%%%%%%%%%%%%%%%%
\begin{equation}
  \begin{aligned}
    & x^0 = 0, && r^0 = u. &
  \end{aligned}
\end{equation}
%%%%%%%%%%%%%%%%%%%%%%%%%%%%%%%%%%%%%%%%%%%%%%%%%%%%%%%%%%%%%%%%%%%%%%%%%%%%%
The iteration ends when $D$ is empty. Excess nodes exist as long as at least
one deficit node exist (guaranteed by the property $\sum_{i \in D} -e_i =
\sum_{i \in E} e_i$ (see~\eqref{eq:QDPFI}). After the last iteration $\eta$,
both $D$ and $E$ are empty -- the network is balanced, i.e. $r$ satisfies node
balance constraints~\eqref{eq:N3KEB}. The maximum circulation is then $z(r^{*})
= z(r^{\eta})$.

The actual algorithm does not maintain $x$, as we only interested in $r$. It's,
thus, just enough to keep track of $r$, $e_i$, $D$ and $E$ only.

\paragraph{Strongly connected components.} To guarantee correctness of the
above algorithm, certain assumptions must be made. For example, the
textbook~\cite{ahuja1993network} provides a list of assumptions, among which
there is one we haven't discussed yet:

{\em Assumption 9.4}~\cite{ahuja1993network}. {\em The network $G$ contains an
uncapacitated directed path (i.e. each arc in the path has infinite capacity)
between every pair of nodes}.

The textbook imposes this condition by adding {\em artificial} arcs $(1, j)$
and $(j, 1)$ for each $j \in N$ and assigning a~large cost and infinite
capacity to each of these arcs (no such arc would appear in a minimum cost
solution [\dots]).

We won't implement arc costs, as we have implicitly assumed $c_{ij} = 1$.
Instead of adding artificial arcs, we'd rather impose similar condition by
removing certain parts of $G$ -- bridges, and leaving only strongly connected
components for further processing.

A~directed graph is called strongly connected if there is a~path in each
direction between each pair of nodes (a~path exist from the first node of the
pair to the second, and another path exists from the second to the first). This
means, that
%%%%%%%%%%%%%%%%%%%%%%%%%%%%%%%%%%%%%%%%%%%%%%%%%%%%%%%%%%%%%%%%%%%%%%%%%%%%%
\begin{itemize}
  \item every node, as well as every arc, of a~strongly connected graph
    participates in at least one cycle.
\end{itemize}
%%%%%%%%%%%%%%%%%%%%%%%%%%%%%%%%%%%%%%%%%%%%%%%%%%%%%%%%%%%%%%%%%%%%%%%%%%%%%
It shall be noted, that the circulation $z(r)$ is a sum of flows along cycles
and, thus, it's enclosed entirely within strongly connected components of $G$.
Eventual bridges between strongly connected components can not contribute to
the circulation (the condensation graph is acyclic). We can thus just remove
bridges from $G$, and solve our problem on the new graph consisting of isolated
strongly connected components only. Identification of bridges for removal can
be done in $\mathcal{O}(|V| + |A|)$ using Tarjan algorithm.

%%\paragraph{Biconnected components.} Visually it looks like, only biconnected
%%components of the whole graph matter. Everything, that's not a~part of
%%a~biconnected component, can not contribute to the circulation. Note, that
%%biconnectivity may mean: (1) a~graph is 2-node connected (no articulation
%%points), but also (2) a graph is 2-arc connected (no bridges). The
%%biconnectivity is defined in context of undirected graphs.
%%
%%\paragraph{Strong stuff.} The terms {\em Strong Articulation Points}, {\em
%%Strong Bridge}, etc. refer specifically to directed graphs. Note, that Tarjan
%%algorithm work with undirected graphs.  See works of Italiano, Firmani, Laura,
%%Orlandi and Santaroni.

%%Suppose we have identified all excess and deficit nodes. In a classical
%%textbook our problem would be modelled by adding two additional nodes $s$
%%(source) and $t$ (sink) to the original network together with arcs $(s,i)$ from
%%the source to each deficit node $i$, and arcs $(j,t)$ from each excess node $j$
%%to $t$. The new arcs $(s,i)$ get assigned capacities $u_{si} = -e_i$, while
%%the new arcs $(j,t)$ have capacities $u_{jt} = e_j$. We then have to find
%%a maximum flow from $s$ to $t$ while maintaining minimal cost \eqref{eq:0NZPS}.

%%\paragraph{Example flow decomposition.}
%%An example flooded network is depicted on figure~\ref{fig:SJ2HQ}. On
%%sub-figure~\ref{fig:AGNK4}, nodes 0, 3 and 6 are deficit, while nodes 5 and 8
%%are excess nodes. The equivalent network (w.r.t internal arc flow) is shown on
%%sub-figure~\ref{fig:FS6JF}. The arcs incident to $s$ have capacities
%%$u_{s0}=1$, $u_{s3}=3$ and $u_{s6}=2$, while arcs incident to $t$ have $u_{5t}
%%= 4$ and $u_{8t} = 2$. These values follow directly from node imbalances on
%%sub-figure~\ref{fig:AGNK4}. The sum of flows over all arcs is
%%$z(u) = \sum{u_{ij}} = 39$.
%%%%%%%%%%%%%%%%%%%%%%%%%%%%%%%%%%%%%%%%%%%%%%%%%%%%%%%%%%%%%%%%%%%%%%%%%%%%%%%
%%\begin{figure}[H]
%%  \centering
%%  \begin{subfigure}[t]{0.42\textwidth}
%%    \centering
%%    \begin{tikzpicture}[>=Stealth]
%%      \graph[
%%        grow right=2cm,
%%        branch down=2cm,
%%        nodes={draw,circle}
%%      ] {
%%        {
%%          0[blue] -> ["4/4"']1 -> ["3/3"'] 2,
%%          {
%%            3[blue] <-["4/4"'] 4 <- ["2/2"'] 5[red],
%%            6[blue] ->["6/6"'] 7 -> ["5/5"'] 8[red]
%%          }
%%        };
%%
%%        3 ->["3/3"] 0;
%%        1 ->["1/1"'] 4;
%%        7 ->["1/1"] 4;
%%        3 ->["4/4"']6;
%%        2 ->["3/3"'] 5;
%%        8 ->["3/3"] 5;
%%      };
%%
%%      \begin{scope}[node distance=1mm,every node/.style={draw,inner sep=2pt},font={\footnotesize}]
%%        % deficit nodes
%%        \node [blue,above right=of 0] {-1};
%%        \node [blue,above right=of 3] {-3};
%%        \node [blue,above right=of 6] {-2};
%%        % excess nodes
%%        \node [red,above right=of 5] {+4};
%%        \node [red,above right=of 8] {+2};
%%        % balanced nodes
%%        \node [above right=of 1] {0};
%%        \node [above right=of 4] {0};
%%        \node [above right=of 7] {0};
%%        \node [above right=of 2] {0};
%%      \end{scope}
%%    \end{tikzpicture}
%%    \caption{Original flooded network with imbalances.}
%%    \label{fig:AGNK4}
%%  \end{subfigure}
%%  \begin{subfigure}[t]{0.55\textwidth}
%%    \centering
%%    \begin{tikzpicture}[>=Stealth]
%%      \graph[
%%        grow right=2cm,
%%        branch down=2cm,
%%        nodes={draw,circle}
%%      ] {
%%        s/$s$[blue] -!- {
%%          0 -> ["4/4"']1 -> ["3/3"']2,
%%          {
%%            3 <- ["4/4"']4 <- ["2/2"']5,
%%            6 -> ["6/6"']7 -> ["5/5"']8
%%          } -!- t/$t$[red]
%%        };
%%
%%        3 -> ["3/3"]0;
%%        1 -> ["1/1"']4;
%%        7 -> ["1/1"]4;
%%        3 -> ["4/4"']6;
%%        2 -> ["3/3"']5;
%%        8 -> ["3/3"]5;
%%
%%        s ->[blue,dashed,"1/1"] 0;
%%        s ->[blue,dashed,"3/3"',out=-60,in=-180] 3;
%%        s ->[blue,dashed,"2/2"',out=-100,in=-180] 6;
%%
%%        5 ->[red,dashed,"4/4"] t;
%%        8 ->[red,dashed,"2/2"',out=0,in=-90] t;
%%      };
%%
%%      \begin{scope}[node distance=1mm,every node/.style={draw,inner sep=2pt},font={\footnotesize}]
%%        % deficit nodes
%%        \node [above right=of 0] {0};
%%        \node [above right=of 3] {0};
%%        \node [above right=of 6] {0};
%%        % excess nodes
%%        \node [above right=of 5] {0};
%%        \node [above right=of 8] {0};
%%        % balanced nodes
%%        \node [above right=of 1] {0};
%%        \node [above right=of 4] {0};
%%        \node [above right=of 7] {0};
%%        \node [above right=of 2] {0};
%%      \end{scope}
%%    \end{tikzpicture}
%%    \caption{Corresponding network with source and sink.}
%%    \label{fig:FS6JF}
%%  \end{subfigure}
%%  \caption{Example flooded network. Arc labels $f_{ij}/u_{ij}$ represent arc
%%           flows $f_{ij}$ over capacities $u_{ij}$. Numbers in rectangles near
%%           to nodes are node imbalances.}
%%  \label{fig:SJ2HQ}
%%\end{figure}
%%%%%%%%%%%%%%%%%%%%%%%%%%%%%%%%%%%%%%%%%%%%%%%%%%%%%%%%%%%%%%%%%%%%%%%%%%%%%%%
%%
%%An example decomposition $r$, $x$, which minimizes the cost $z(x)$, is given on
%%figure~\ref{fig:W6D11}. From the sub-figure~\ref{fig:3SOTE} -- the cost $z(x) =
%%\sum{x_{ij}} = 19$ (not counting the arcs incident to $s$ and $t$), and thus,
%%the maximum circulation is~$z(r) = z(u) - z(x) = 39 - 19 = 20$. The
%%decomposition is a~path and cycle flow decomposition~\cite{ahuja1993network}
%%with the following components:
%%%%%%%%%%%%%%%%%%%%%%%%%%%%%%%%%%%%%%%%%%%%%%%%%%%%%%%%%%%%%%%%%%%%%%%%%%%%%%%
%%\begin{itemize}
%%  \item $x$ is a~flow consisting of path flows (from shortest to longest):
%%
%%      $2$ units sent through path $s-6-7-8-t$ of length $2$,
%%      $\bullet$~$1$ unit send through path $s-0-1-2-5-t$ of length $3$, and
%%      $\bullet$~$3$ units sent through path $s-3-6-7-8-5-t$  of length $4$, the
%%      total cost is $z(x) = 2\cdot2 + 1\cdot3 + 3\cdot 4 = 19$. The path
%%      lengths used for calculations do not include arcs incident to $s$ and $t$.
%%
%%  \item $r$ is a~flow consisting of cycle flows:
%%
%%      $1$~unit sent over cycle $0-1-4-3-0$ (cycle length 4),
%%      $\bullet$~$2$ units over $0-1-2-5-4-3-0$ (cycle length 6),
%%      $\bullet$~$1$ unit over $6-7-4-3-6$ (cycle length 4). The circulation is
%%      $1\cdot 4 + 2\cdot 6 + 1\cdot 4 = 20$.
%%\end{itemize}
%%%%%%%%%%%%%%%%%%%%%%%%%%%%%%%%%%%%%%%%%%%%%%%%%%%%%%%%%%%%%%%%%%%%%%%%%%%%%%%
%%%%%%%%%%%%%%%%%%%%%%%%%%%%%%%%%%%%%%%%%%%%%%%%%%%%%%%%%%%%%%%%%%%%%%%%%%%%%%%
%%\begin{figure}[H]
%%  \centering
%%  \begin{subfigure}[t]{0.42\textwidth}
%%    \centering
%%    \begin{tikzpicture}[>=Stealth]
%%      \graph[
%%        grow right=2cm,
%%        branch down=2cm,
%%        nodes={draw,circle}
%%      ] {
%%        {
%%          0 -> ["3/4"']1 -> ["2/3"'] 2,
%%          {
%%            3 <-["4/4"'] 4 <- ["2/2"'] 5,
%%            6 ->["1/6"'] 7 -> ["0/5"'] 8
%%          }
%%        };
%%
%%        3 ->["3/3"] 0;
%%        1 ->["1/1"'] 4;
%%        7 ->["1/1"] 4;
%%        3 ->["1/4"']6;
%%        2 ->["2/3"'] 5;
%%        8 ->["0/3"] 5;
%%      };
%%
%%      \begin{scope}[node distance=1mm,every node/.style={draw,inner sep=2pt},font={\footnotesize}]
%%        % deficit nodes
%%        \node [above right=of 0] {0};
%%        \node [above right=of 3] {0};
%%        \node [above right=of 6] {0};
%%        % excess nodes
%%        \node [above right=of 5] {0};
%%        \node [above right=of 8] {0};
%%        % balanced nodes
%%        \node [above right=of 1] {0};
%%        \node [above right=of 4] {0};
%%        \node [above right=of 7] {0};
%%        \node [above right=of 2] {0};
%%      \end{scope}
%%    \end{tikzpicture}
%%    \caption{Circular flow $r = \left\lbrace r_{ij}\right\rbrace$.}
%%    \label{fig:8U00Z}
%%  \end{subfigure}
%%  \begin{subfigure}[t]{0.55\textwidth}
%%    \centering
%%    \begin{tikzpicture}[>=Stealth]
%%      \graph[
%%        grow right=2cm,
%%        branch down=2cm,
%%        nodes={draw,circle}
%%      ] {
%%        s/$s$[blue] -!- {
%%          0 -> ["1/4"']1 -> ["1/3"']2,
%%          {
%%            3 <- ["0/4"']4 <- ["0/2"']5,
%%            6 -> ["5/6"']7 -> ["5/5"']8
%%          } -!- t/$t$[red]
%%        };
%%
%%        3 -> ["0/3"]0;
%%        1 -> ["0/1"']4;
%%        7 -> ["0/1"]4;
%%        3 -> ["3/4"']6;
%%        2 -> ["1/3"']5;
%%        8 -> ["3/3"]5;
%%
%%        s ->[blue,dashed,"1/1"] 0;
%%        s ->[blue,dashed,"3/3"',out=-60,in=-180] 3;
%%        s ->[blue,dashed,"2/2"',out=-100,in=-180] 6;
%%
%%        5 ->[red,dashed,"4/4"] t;
%%        8 ->[red,dashed,"2/2"',out=0,in=-90] t;
%%      };
%%
%%      \begin{scope}[node distance=1mm,every node/.style={draw,inner sep=2pt},font={\footnotesize}]
%%        % deficit nodes
%%        \node [above right=of 0] {0};
%%        \node [above right=of 3] {0};
%%        \node [above right=of 6] {0};
%%        % excess nodes
%%        \node [above right=of 5] {0};
%%        \node [above right=of 8] {0};
%%        % balanced nodes
%%        \node [above right=of 1] {0};
%%        \node [above right=of 4] {0};
%%        \node [above right=of 7] {0};
%%        \node [above right=of 2] {0};
%%      \end{scope}
%%    \end{tikzpicture}
%%    \caption{Path flow $x = \left\lbrace x_{ij}\right\rbrace$.}
%%    \label{fig:3SOTE}
%%  \end{subfigure}
%%  \caption{One of the optimal solutions for the network from figure \ref{fig:SJ2HQ}.}
%%  \label{fig:W6D11}
%%\end{figure}
%%%%%%%%%%%%%%%%%%%%%%%%%%%%%%%%%%%%%%%%%%%%%%%%%%%%%%%%%%%%%%%%%%%%%%%%%%%%%%%

%%Let's consider certain corner cases, before proceeding to the actual solution.
%%
%%\paragraph{Loops.}
%%Circulation is only possible along cycles in the graph. The simplest possible
%%cycle is a~loop. It corresponds to a request to stay in same building $i$. All
%%such requests can be accepted at hand, as they satisfy both
%%constraints~\eqref{eq:YA13L}. The maximum number of possible
%%transfers along a~loop is always $r_{ii} = u_{ii}$.
%%%%%%%%%%%%%%%%%%%%%%%%%%%%%%%%%%%%%%%%%%%%%%%%%%%%%%%%%%%%%%%%%%%%%%%%%%%%%%%
%%\begin{figure}[htbp]
%%  \centering
%%  \begin{tikzpicture}
%%    \graph[nodes={draw,circle}, clockwise=2] {
%%      i; j;
%%      i -> [out=-45, in=45, looseness=7, "$u_{ii}$"] i;
%%      i -> [bend right, "$u_{ij}$"] j;
%%    };
%%  \end{tikzpicture}
%%  \caption{Example of a~loop in the graph. The number of requests in the loop
%%  is $u_{ii}$. The maximum possible number of ``transfers'' is also $u_{ii}$
%%  (all employees are allowed to stay in their buildings).}
%%  \label{fig:4B7KI}
%%\end{figure}
%%%%%%%%%%%%%%%%%%%%%%%%%%%%%%%%%%%%%%%%%%%%%%%%%%%%%%%%%%%%%%%%%%%%%%%%%%%%%%%
%%Technically, we can just sum up all flows along existing loops and just remove
%%them from the graph before proceeding with actual optimization or (better) sum
%%up all requests of type $(i, i)$ during graph construction without adding these
%%loops to the graph.
%%
%%\paragraph{Pure sources and sinks (``terminal'' nodes).}
%%Nodes that aren't part of any cycles can be removed as well
%%(figure~\ref{fig:57L58}), as they can't contribute to the circulation.
%%%%%%%%%%%%%%%%%%%%%%%%%%%%%%%%%%%%%%%%%%%%%%%%%%%%%%%%%%%%%%%%%%%%%%%%%%%%%%%
%%\begin{figure}[htbp]
%%  \centering
%%  \begin{tikzpicture}
%%    \matrix[nodes in empty cells] {
%%      & \node[anchor=north]{\tikz{\graph[nodes={draw,circle}] {
%%        0 -> 1[x=1] -> 2 -> 0;
%%        0 -> 3[r=1] ->[red,dashed] 4[red,dashed,r=1.5];
%%        2 -> 3;
%%        2 ->[red,dashed] 4;
%%        5[r=-0.5,x=0.75,red,dashed] ->[red,dashed] 3;
%%      };}};
%%
%%      & \node[anchor=north,minimum height=3cm]{$\implies$}; &
%%
%%      \node[anchor=north]{\tikz{\graph[nodes={draw,circle}] {
%%        0 -> 1[x=1] -> 2 -> 0;
%%        0 ->[red,dashed] 3[r=1,red,dashed];
%%        2 ->[red,dashed] 3;
%%      };}};
%%
%%      & \node[anchor=north,minimum height=3cm]{$\implies$}; &
%%
%%      \node[anchor=north]{\tikz{\graph[nodes={draw,circle}] {
%%        0 -> 1[x=1] -> 2 -> 0;
%%      };}}; & \\
%%
%%      & \node{original}; & & & & \node{reduced}; & \\
%%    };
%%  \end{tikzpicture}
%%  \caption{Eliminating ``terminal'' nodes. Initially, node $5$ has
%%           indegree=0, and $4$ has outdegree=$0$. After removal of $4$ and $5$,
%%           node $3$ becomes terminal (outdegree=$0$). Removing node $3$
%%           leaves only nodes belonging to a~cycle.}
%%  \label{fig:57L58}
%%\end{figure}
%%%%%%%%%%%%%%%%%%%%%%%%%%%%%%%%%%%%%%%%%%%%%%%%%%%%%%%%%%%%%%%%%%%%%%%%%%%%%%%
%%This can be done by iterative removal of nodes whose
%%indegree\footnote{The terms {\em indegree/outdegree} denote the number of
%%incoming/outgoing arcs for a node.} or outdegree is zero. Note, that this
%%includes isolated nodes as well. The removal of ``terminal'' nodes is an
%%iterative process, as removing one ``terminal'' may create new ``terminals''.

%%\paragraph{Problem formulation.}
%%To find the maximum circulation, we may initially ``flood'' the network by
%%setting $r_{ij} = u_{ij}$ and then eliminate non-circular flow component. What
%%remains after this operation, is always a~circulation (not necessarily
%%maximum). The concept is based on the flow decomposition
%%theorem~\cite{ahuja1993network} which states that every non-negative arc flow
%%$r$ can be represented as a path and cycle flow (though not necessarily
%%uniquely). This way, the total flow $\sum{u_{ij}}$ in the ``flooded'' network
%%may be decomposed onto non-circular and circular flows as follows
%%%%%%%%%%%%%%%%%%%%%%%%%%%%%%%%%%%%%%%%%%%%%%%%%%%%%%%%%%%%%%%%%%%%%%%%%%%%%%%
%%\begin{equation}
%%  \sum_{(i,j) \in A} u_{ij} =
%%  \underbrace{\sum_{P \in \mathcal{P}} f(P) |P|}_{\text{Non-circular flow}} +
%%  \underbrace{\sum_{W \in \mathcal{W}} {f(W) |W|}}_{\text{Circular flow}} = \mathrm{const}.
%%\label{eq:XW7AC}
%%\end{equation}
%%%%%%%%%%%%%%%%%%%%%%%%%%%%%%%%%%%%%%%%%%%%%%%%%%%%%%%%%%%%%%%%%%%%%%%%%%%%%%%
%%The pair $\left(\mathcal{P}, \mathcal{W}\right) \in \mathcal{D}$
%%in~\eqref{eq:XW7AC} represents a~flow decomposition (one of many possible).
%%The~decomposition is defined by a~set $\mathcal{P}$ of path flows $P$, and a~set
%%$\mathcal{W}$ of cycle flows $W$. Each path flow $P$ is specified by a~path
%%(sequence of arcs) in graph $G$ and a~flow $f(P)$ along the path. Similarly,
%%the cycle flow $W$ is specified by a~cycle in graph $G$ and a~flow $f(W)$ along
%%the cycle. $|P|$ and $|W|$ are, correspondingly, path and cycle lengths (number
%%of arcs). Among all possible decompositions $\mathcal{D}$, there is at least
%%one, which maximizes the resulting circulation (and, naturally, minimizes the
%%non-circular flow in~\eqref{eq:XW7AC}).
%%
%%By setting $r_{ij} = u_{ij}$ for all $(i,j) \in A$ we'll impose imbalances
%%$e_i$ in nodes
%%%%%%%%%%%%%%%%%%%%%%%%%%%%%%%%%%%%%%%%%%%%%%%%%%%%%%%%%%%%%%%%%%%%%%%%%%%%%%%
%%\begin{equation}
%%  \begin{aligned}
%%    e_i = \sum_{j: (j,i) \in A} r_{ji} - \sum_{j: (i,j) \in A} r_{ij}, &
%%    && \text{ for all } i \in V.
%%  \end{aligned}
%%  \label{eq:6NYTI}
%%\end{equation}
%%%%%%%%%%%%%%%%%%%%%%%%%%%%%%%%%%%%%%%%%%%%%%%%%%%%%%%%%%%%%%%%%%%%%%%%%%%%%%%
%%In such a setup, we can identify {\em deficit} nodes having $e_i < 0$,
%%{\em excess} nodes having $e_i > 0$ and search for directed paths from deficit
%%to excess nodes. The flow decomposition theorem provides us with an algorithmic
%%way of identifying path and cycle flows. It's summarized as follows.

%%\paragraph{Identifying a path flow.}
%%Suppose that $i_0$ is a deficit node. Then some arc $(i_0, i_1)$ carries a
%%positive flow. If $i_1$ is an excess node, we stop; otherwise, the mass balance
%%constraint of node $i_1$ implies that some other arc $(i_1, i_2)$ carries
%%positive flow. We repeat this until we encounder an excess node, or we revisit
%%a~previously examined node. In the former case, we obtain a directed path $P$
%%from the deficit node $i_0$ to some excess node $i_k$. In the later case we
%%obtain a directed cycle $W$. If we obtain a~directed path $P$, we let $f(P) =
%%\min{\left\lbrace-e(i_0), e(i_k), \min{\left\lbrace r_{ij}: (i,j) \in P
%%\right\rbrace}\right\rbrace}$.
%%
%%%%and redefine $e(i_0) := e(i_0) + f(P)$, $e(i_k) :=
%%%%e(i_k) - f(P)$ and $r_{ij} := r_{ij} - f(P)$ for each arc $(i,j) \in P$. If we
%%%%obtain a directed cycle $W$, we let $f(W) = \min{\left\lbrace r_{ij}:(i,j) \in
%%%%W\right\rbrace}$ and redefine $r_{ij} = r_{ij} - f(W)$ for each arc $(i,j) \in
%%%%W$. We also add the value $f(W) \cdot |W|$ to the final result, where $|W|$ is
%%%%the number of arcs in the cycle $W$.
%%
%%We repeat this process with the redefined problem until all node imbalances
%%$e_i$ are zero. We then add all the the remaining arc flows $r_{ij}$ to the
%%final result (the remaining flow is a circulation).
%%
%%The time complexity of the above algorithm is said to be $O(nm)$~\cite{ahuja1993network}.
%%
%% For~a 2-node cycle $i \to j \to i$ (figure~\ref{fig:DRV1T}), the maximum number
%% $r_{ij}$ of possible transfers is
%% %%%%%%%%%%%%%%%%%%%%%%%%%%%%%%%%%%%%%%%%%%%%%%%%%%%%%%%%%%%%%%%%%%%%%%%%%%%%%
%% \begin{equation}
%%   r_{ij} = 2 u_{\min} = 2 \min{(u_{ij}, u_{ji})}.
%%   \label{eq:FMMLO}
%% \end{equation}
%% %%%%%%%%%%%%%%%%%%%%%%%%%%%%%%%%%%%%%%%%%%%%%%%%%%%%%%%%%%%%%%%%%%%%%%%%%%%%%
%% %%%%%%%%%%%%%%%%%%%%%%%%%%%%%%%%%%%%%%%%%%%%%%%%%%%%%%%%%%%%%%%%%%%%%%%%%%%%%
%% \begin{figure}[htbp]
%%   \centering
%%   \begin{tikzpicture}
%%     \matrix {
%%       \graph[nodes={draw,circle}, clockwise=2] {
%%         i; j;
%%         i -> [bend left, "$u_{ij}$"] j ->[bend left,"$u_{ji}$"] i;
%%       }; & \node {$\implies$}; & \graph[nodes={draw,circle}, clockwise=2] {
%%         i; j;
%%         i -> [bend left, "$u_{\min}$"] j ->[bend left,"$u_{\min}$"] i;
%%       }; \\
%%       \node {requested}; & & \node {possible}; \\
%%     };
%%   \end{tikzpicture}
%%   \caption{Example two-node cycle. The number of requests from $i$ to $j$ is
%%   $u_{ij}$, from $j$ to $i$ is $u_{ji}$. The maximum possible number of mutual
%%   transfers depends on the $u_{\min} = \min{(u_{ij}, u_{ji})}$.}
%%   \label{fig:DRV1T}
%% \end{figure}
%% %%%%%%%%%%%%%%%%%%%%%%%%%%%%%%%%%%%%%%%%%%%%%%%%%%%%%%%%%%%%%%%%%%%%%%%%%%%%%
%% Note, that if we remove all the edges that comprise the cycle, still some edges
%% may be left on the graph (rejected requests). In this case, the number of
%% rejected requests is $u_{ij} + u_{ji} - 2 u_{\min}$. Finding the number of
%% achievable transfers is equivalent to finding the minimum number of requests
%% that must be rejected.
%%
%% For a~three-node/three-edge cycle, as in figure~\ref{fig:U5GUM}, the maximum
%% possible number of transfers is
%% %%%%%%%%%%%%%%%%%%%%%%%%%%%%%%%%%%%%%%%%%%%%%%%%%%%%%%%%%%%%%%%%%%%%%%%%%%%%%
%% \begin{equation}
%%   r_{ijk} = 3 u_{\min} = 3 \min{(u_{ij}, u_{jk}, u_{ki})}
%%  \label{eq:S95QR}
%% \end{equation}
%% %%%%%%%%%%%%%%%%%%%%%%%%%%%%%%%%%%%%%%%%%%%%%%%%%%%%%%%%%%%%%%%%%%%%%%%%%%%%%
%% %%%%%%%%%%%%%%%%%%%%%%%%%%%%%%%%%%%%%%%%%%%%%%%%%%%%%%%%%%%%%%%%%%%%%%%%%%%%%
%% \begin{figure}[htbp]
%%   \centering
%%   \begin{tikzpicture}
%%     \matrix {
%%       \graph[nodes={draw,circle}, clockwise=3] {
%%         i; j; k;
%%         i ->[bend left, "$u_{ij}$"] j ->[bend left,"$u_{jk}$"]  k ->[bend left, "$u_{ki}$"] i;
%%       }; & \node {$\implies$}; & \graph[nodes={draw,circle}, clockwise=3] {
%%         i; j; k;
%%         i ->[bend left, "$u_{\min}$"] j ->[bend left,"$u_{\min}$"]  k ->[bend left, "$u_{\min}$"] i;
%%       }; \\
%%       \node {requested}; & & \node {possible}; \\
%%     };
%%   \end{tikzpicture}
%%   \caption{Example three-node cycle}
%%   \label{fig:U5GUM}
%% \end{figure}
%% %%%%%%%%%%%%%%%%%%%%%%%%%%%%%%%%%%%%%%%%%%%%%%%%%%%%%%%%%%%%%%%%%%%%%%%%%%%%%
%%
%% In general, the~number of possible transfers along an~isolated
%% $K$-node/$K$-edge cycle is a~multiple of~$K$, and has the form
%% $t = K u_{\min}$, where $u_{\min}$ is the smallest edge multiplicity along the
%% path.
%%
%% %%We guess, that for a~$K$-node cycle $\mathcal{C}$, the maximum possible number
%% %%of transfers is
%% %%%%%%%%%%%%%%%%%%%%%%%%%%%%%%%%%%%%%%%%%%%%%%%%%%%%%%%%%%%%%%%%%%%%%%%%%%%%%%%
%% %%\begin{equation}
%% %%  m_{\mathcal{C}} = K \min_{e_{ij} \in \mathcal{C}}{\{e_{ij}\}}
%% %%  \label{eq:63VA3}
%% %%\end{equation}
%% %%%%%%%%%%%%%%%%%%%%%%%%%%%%%%%%%%%%%%%%%%%%%%%%%%%%%%%%%%%%%%%%%%%%%%%%%%%%%%%
%%
%% Isolated cycles are simple. In reality, however, cycles can partially overlap.
%% Consider a~three node graph with two different cycles as shown on
%% figure~\ref{fig:XDUOL}. Total number of requests is $u_{ij} + u_{jk} + u_{ki} +
%% u_{kj}$, but there are constraints limiting the possible number of transfers.
%% %%%%%%%%%%%%%%%%%%%%%%%%%%%%%%%%%%%%%%%%%%%%%%%%%%%%%%%%%%%%%%%%%%%%%%%%%%%%%
%% \begin{figure}[htbp]
%%   \centering
%%   \begin{tikzpicture}
%%     \graph[nodes={draw,circle}, clockwise=3]{
%%       i; j; k;
%%       i ->[bend left, "$u_{ij}$"] j
%%         ->[bend left, "$u_{jk}$"] k
%%         ->[bend left, "$u_{ki}$"] i;
%%       k ->[bend left, "$u_{kj}$"] j;
%%     };
%%   \end{tikzpicture}
%%   \caption{Example three-node graph with two cycles}
%%   \label{fig:XDUOL}
%% \end{figure}
%% %%%%%%%%%%%%%%%%%%%%%%%%%%%%%%%%%%%%%%%%%%%%%%%%%%%%%%%%%%%%%%%%%%%%%%%%%%%%%
%% The edge $j \to k$ is shared between cycles $i \to j \to k \to i$ and $j \to k
%% \to j$. From the other perspective, the edge $j \to k$ contributes to either
%% the $i \to j \to k \to i$ or $j \to k \to j$, or to both.

\bibliographystyle{unsrtnat}
\bibliography{leetcode}

\end{document}


% vim: set syntax=tex tabstop=2 shiftwidth=2 expandtab spell spelllang=en:
