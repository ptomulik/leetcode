\documentclass[paper=a4,parskip=half,DIV=12]{leetcode}

\usepackage[T1]{fontenc}
\usepackage[utf8]{inputenc}
\usepackage{courier}
\usepackage{tgtermes,newtxtext,newtxmath}
\usepackage[pdftex,colorlinks,allcolors=blue]{hyperref}
\usepackage{natbib}

\usepackage{amsmath,amsfonts}
\usepackage{tikz}
\usetikzlibrary{arrows,3d,patterns,calc}

\setcitestyle{numbers,square,comma}

\begin{document}

\serietitle{LeetCode contests solutions}
\title{1601. Maximum Number of Achievable Transfer Requests~\cite{leetcode:1601}}
\subtitle{}
\author{Paweł Tomulik}
\date{2024-06-20}
\maketitle

\section{Description}
\label{sec:description}

We have $N$ buildings numbered from $0$ to $N - 1$. Each building has a number
of employees. It's transfer season, and some employees want to change the
building they reside in.

You are given an array \texttt{requests} where \texttt{requests[i] = [from$_i$,
to$_i$]} represents an employee's request to transfer from building
\texttt{from$_i$} to building \texttt{to$_i$}.

All buildings are full, so a list of requests is achievable only if for each
building, the net change in employee transfers is zero. This means the number
of employees leaving is equal to the number of employees moving in. For example
if $N = 3$ and two employees are leaving building $0$, one is leaving building
$1$, and one is leaving building $2$, there should be two employees moving to
building $0$, one employee moving to building $1$, and one employee moving to
building $2$.

Return the \textbf{maximum number of achievable requests}.

\section{Solution}
\label{sec:solution}

\subsection{Complexity}
\label{sec:complexity}

\subsection{Solution description}
\label{sec:solution-description}

This problem can be modelled with a~directed graph. Graph nodes represent
buildings, edges correspond to transfer requests. Edges will have integer
weights assigned:
%%%%%%%%%%%%%%%%%%%%%%%%%%%%%%%%%%%%%%%%%%%%%%%%%%%%%%%%%%%%%%%%%%%%%%%%%%%%%
\begin{equation}
    e_{ij}, \; i, j \in \{ 0, 1, \dots N-1 \}, \; i \neq j
    \label{eq:49EZ8}
\end{equation}
%%%%%%%%%%%%%%%%%%%%%%%%%%%%%%%%%%%%%%%%%%%%%%%%%%%%%%%%%%%%%%%%%%%%%%%%%%%%%
where $e_{ij}$ is a~number of transfer requests from building $i$ to $j$. If
there are no transfers from $i$ to $j$, then there is no edge $i \to j$.

Consider a pair of nodes $(i, j)$ that form cycle $i \to j \to i$. The maximum
number of possible transfers between the corresponding buildings is then
%%%%%%%%%%%%%%%%%%%%%%%%%%%%%%%%%%%%%%%%%%%%%%%%%%%%%%%%%%%%%%%%%%%%%%%%%%%%%
\begin{equation}
  m_2 = 2 \min{(e_{ij}, e_{ji})}
  \label{eq:FMMLO}
\end{equation}
%%%%%%%%%%%%%%%%%%%%%%%%%%%%%%%%%%%%%%%%%%%%%%%%%%%%%%%%%%%%%%%%%%%%%%%%%%%%%
Going further, for a~three-node cycle $(i, j, k)$, we have
%%%%%%%%%%%%%%%%%%%%%%%%%%%%%%%%%%%%%%%%%%%%%%%%%%%%%%%%%%%%%%%%%%%%%%%%%%%%%
\begin{equation}
  m_3 = 3 \min{(e_{ij}, e_{jk}, e_{ki})}
  \label{eq:FMMLO}
\end{equation}
%%%%%%%%%%%%%%%%%%%%%%%%%%%%%%%%%%%%%%%%%%%%%%%%%%%%%%%%%%%%%%%%%%%%%%%%%%%%%


\bibliographystyle{unsrtnat}
\bibliography{leetcode}

\end{document}

%\label{??:63VA3}
%\label{??:W6D11}
%\label{??:SJ2HQ}
%\label{??:0NZPS}
%\label{??:A20W8}
%\label{??:6M1IM}
%\label{??:KSU4J}
%\label{??:KRXUE}
%\label{??:P2GTP}
%\label{??:I4AB9}
%\label{??:D2Y3O}
%\label{??:XW7AC}
%\label{??:6NYTI}
%\label{??:57L58}
%\label{??:HP3C4}
%\label{??:UDSSB}
%\label{??:YA13L}
%\label{??:4B7KI}
%\label{??:XDUOL}
%\label{??:U5GUM}
%\label{??:S95QR}
%\label{??:DRV1T}

% vim: set syntax=tex tabstop=2 shiftwidth=2 expandtab spell spelllang=en:
